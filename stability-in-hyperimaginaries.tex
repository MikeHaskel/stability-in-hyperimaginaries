\documentclass{article}


%%% Packages

% AMS mathematics
\usepackage{amsmath,amsthm,amssymb}

% Make in-document references clickable
\usepackage{hyperref}

% Custom package for the \anchor symbol used to denote ``independence'' in model theory
\usepackage{anchor}


%%% Theorem environments

\newtheorem{theorem}{Theorem}[section]

\theoremstyle{definition}
\newtheorem{definition}[theorem]{Definition}


%%% Custom commands

\newcommand{\defterm}{\emph}


%%% Header material

\title{Stability in Hyperimaginary Sorts}
\author{Mike Haskel}
\date{Spring 2017}


%%% Content

\begin{document}
\maketitle

\begin{abstract}
  In this paper, we develop basic notions of local stability theory in hyperimaginary sorts.
\end{abstract}

\section{Hyperimaginaries}

Throughout this paper, let $T$ be a first-order $L$-theory, not necessarily complete. Definable and type-definable sets will be over $\emptyset$ unless otherwise specified. Letters $X$, $Y$, etc.\ will denote type-definable sets of infinite tuples; $x$, $y$, etc.\ will denote tuples of variables in sorts $X$, $Y$, etc.; $\varphi(x,y)$, etc.\ will denote relatively definable subsets of $X \times Y$, etc.; $\Phi(x,y)$, etc.\ will denote type-definable subsets of $X \times Y$, etc.

\begin{definition}
  Given $X$, let $x_1 E x_2$ be a type-definable subset of $X \times X$. Say $X$ is an \defterm{equivalence relation} if in every model of $T$ it is interpreted as an equivalence relation. A \defterm{hyperimaginary sort} is given by the pair $(X,E)$ and written $X/E$. Its interpretation in a model $M$ of $T$, written $X/E(M)$, is the quotient set $X(M)/E(M)$.
\end{definition}

\end{document}
